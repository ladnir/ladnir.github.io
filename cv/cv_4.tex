%%%%%%%%%%%%%%%%%%%%%%%%%%%%%%%%%%%%%%%%%
% Medium Length Professional CV
% LaTeX Template
% Version 2.0 (8/5/13)
%
% This template has been downloaded from:
% http://www.LaTeXTemplates.com
%
% Original author:
% Trey Hunner (http://www.treyhunner.com/)
%
% Important note:
% This template requires the resume.cls file to be in the same directory as the
% .tex file. The resume.cls file provides the resume style used for structuring the
% document.
%
%%%%%%%%%%%%%%%%%%%%%%%%%%%%%%%%%%%%%%%%%

%----------------------------------------------------------------------------------------
%	PACKAGES AND OTHER DOCUMENT CONFIGURATIONS
%----------------------------------------------------------------------------------------

\documentclass{resume} % Use the custom resume.cls style

\usepackage[left=0.75in,top=0.6in,right=0.75in,bottom=0.6in]{geometry} % Document margins
\usepackage{url,enumitem, xcolor}

\name{Peter Rindal} % Your name
\address{(509)~$\cdot$~520~$\cdot$~8701 \\ PeterRindal@gmail.com} % Your address
\address{\url{ladnir.github.io}} % Your secondary addess (optional)
\address{} % Your phone number and email

\begin{document}

%----------------------------------------------------------------------------------------
%	EDUCATION SECTION
%----------------------------------------------------------------------------------------

\begin{rSection}{Education}

{\bf Ph.D. in Computer Science} \hfill {\em January 2015 --- Est. Sep. 2018} \\ 
Oregon State University, Corvallis\smallskip \\
Overall GPA: 3.8\\ 


{\bf M.S. in Computer Science} \hfill {\em January 2015 --- Sep. 2017} \\ 
Oregon State University, Corvallis\smallskip \\
Overall GPA: 3.8\\ 


{\bf B.S. in Computer Science} \hfill {\em September 2010 --- June 2014} \\ 
Oregon State University, Corvallis\smallskip \\
Overall GPA: 3.65\\ 

\end{rSection}


%----------------------------------------------------------------------------------------
%	Research Interests
%----------------------------------------------------------------------------------------

\begin{rSection}{Research Interests}
	
My primary interest is the development of efficient methods for computing on encrypted data. Most notably has been the development of highly optimized protocols for performing  Private Set Intersection for both malicious \& semi-honest adversaries. I have also worked on multi-party authenticated encryption, and several projects combining machine learning, differential privacy and secure computation.
	
\end{rSection}

%----------------------------------------------------------------------------------------
%	WORK EXPERIENCE SECTION
%----------------------------------------------------------------------------------------

\begin{rSection}{Employment}

{\bf Visa Research} \hfill {August 2018 --- present}\\
{\emph{Staff Research Scientist } \hfill {San Francisco, CA}}

{\bf Oregon State University} \hfill {January 2015 --- August 2018}\\
{\emph{Graduate Research Assistant} \hfill {Corvallis, OR}}

{\bf Visa Research} \hfill {June 2017 --- September 2017}\\
{\emph{Security Research Intern} \hfill {Palo Alto, CA}}

{\bf Microsoft Research} \hfill {June 2016 --- September 2016}\\
{\emph{Security Research Intern} \hfill {Redmond, WA}}

{\bf Microsoft Research} \hfill {January 2016 --- March 2016}\\
{\emph{Security Research Intern} \hfill {Redmond, WA}}

{\bf Digimarc} \hfill {June 2014 --- December 2014}\\
{\emph{Software Developer Intern} \hfill {Portland, OR}}

{\bf Boeing Company} \hfill {March 2013 --- September 2013}\\
{\emph{Software Developer Intern} \hfill {Portland, OR}}

%\begin{rSubsection}{Visa Research}{June 2017 - September 2017}{Security Research Intern}{Palo Alto, CA}
%	\item Developed efficient protocols for training machine learning models on encrypted data with security against semi-honest and malicious adversaries.
%	\item Proposed a new definitions and constructions for distributed CPA, CCA and Authenticated Encryption where the key is securely split amongst several parties.
%\end{rSubsection}

%------------------------------------------------

\end{rSection}

%----------------------------------------------------------------------------------------
%	Publications
%----------------------------------------------------------------------------------------

\begin{rSection}{Publications}
\hfill {\scriptsize \textcolor{gray}{\emph{Note: the standard convention in this discipline is to list authors alphabetically.}}}

Peer-reviewed conference publications:
\begin{enumerate}[label=C\arabic* --]
	
	\item Peter Rindal,  Phillipp Schoppmann. \emph{VOLE-PSI: Fast OPRF and Circuit-PSI from Vector-OLE.} In \emph{EUROCRYPT: IACR International Cryptology Conference 2017.}
	
	\item Saikrishna Badrinarayanan, Peihan Miao, Peter Rindal. \emph{Multi-Party Threshold Private Set Intersection with Sublinear Communication.} In \emph{PKC: Practice and Theory of Public-Key Cryptography 2021.}
	
	\item Payman Mohassel, Peter Rindal and Mike Rosulek. \emph{Fast Database Joins and PSI for Secret Shared Data.} In \emph{CCS: ACM Conference on Computer and Communications Security 2020.}
	
	\item Shashank Agrawal, Saikrishna Badrinarayanan, Pratyay Mukherjee, Peter Rindal.
	\emph{Game-Set-MATCH: Using Mobile Devices for Seamless External-Facing Biometric Matching}. In \emph{CCS: ACM Conference on Computer and Communications Security  2020.}
	
	\item Elette Boyle, Geoffroy Couteau, Niv Gilboa, Yuval Ishai, Lisa Kohl, Peter Rindal, Peter Scholl.
	\emph{Efficient Two-Round OT Extension and Silent Non-Interactive Secure Computation.} In \emph{CCS: ACM Conference on Computer and Communications Security   2019.}
	
	\item Daniel Masny, Peter Rindal. \emph{Endemic Oblivious Transfer.} In \emph{CCS: ACM Conference on Computer and Communications Security 2019}
	
	\item Adam Groce, Peter Rindal and Mike Rosulek. \emph{Cheaper Private Set Intersection  via Differentially Private Leakage.}  In \emph{PETS: Privacy Enhancing Technologies Symposium 2019.}
	
	\item Daniel Demmler, Peter Rindal, Mike Rosulek and Ni Trieu. \emph{PIR-PSI: Scaling Private Contact Discovery.}  In \emph{PETS: Privacy Enhancing Technologies Symposium 2018.}
	
	\item Payman Mohassel and Peter Rindal. \emph{$ABY^3$: A Mixed Protocol Framework for Machine Learning.} In \emph{CCS: ACM Conference on Computer and Communications Security 2018.}
	
	\item Shashank Agrawal, Payman Mohassel, Pratyay Mukherjee and Peter Rindal. \emph{Threshold Authenticated Encryption}. In \emph{CCS: ACM Conference on Computer and Communications Security 2018.}	
	
	\item Hao Chen, Kim Laine and Peter Rindal. \emph{Labed-PSI: Improved Unbalanced Private Set Intersection with Fully Homomorphic Encryption.} In \emph{CCS: ACM Conference on Computer and Communications Security 2018.}
	
	
	\item Peter Rindal and Mike Rosulek. \emph{Faster Malicious 2-party Secure Computation with Online/Offline Dual Execution.} In \emph{USENIX Security Symposium 2016.}
	
	\item Gizem Cetin, Hao Chen, Kim Laine, Kristin Lauter, Peter Rindal and Yuhou Xia. \emph{Private Queries on Encrypted Genomic Data.} In \emph{BMC Medical Genomics:  iDASH Privacy and Security Workshop 2016.}
		
	\item Peter Rindal and Mike Rosulek. \emph{Improved Private Set Intersection against Malicious Adversaries.} In \emph{EUROCRYPT: IACR International Cryptology Conference 2017.}
	
	\item Hao Chen, Kim Laine and Peter Rindal. \emph{Fast Private Set Intersection from Homomorphic Encryption.} In \emph{CCS: ACM Conference on Computer and Communications Security 2017.}
	
	\item Peter Rindal and Mike Rosulek. \emph{Malicious-Secure Private Set Intersection via Dual Execution.} In \emph{CCS: ACM Conference on Computer and Communications Security 2017.}
\end{enumerate}

\bigskip 

\bigskip 
Informal publications:
\begin{enumerate}[label=I\arabic* --]
	
	\item Ran Gilad-Bachrach, Kim Laine, Kristin Lauter, Peter Rindal and Mike Rosulek. \emph{Secure Data Exchange: A Marketplace in the Cloud.} In \emph{IACR ePrint 2016.}
		
	\item Peter Rindal and Roberto Trifiletti. \emph{SplitCommit: Implementing and Analyzing Homomorphic UC Commitments.} In \emph{IACR ePrint 2017.}
	
	\item Melissa Chase, Ran Gilad-Bachrach, Kim Laine, Kristin Lauter and Peter Rindal. \emph{Private Collaborative Neural Network Learning.} In \emph{IACR ePrint 2017.}
\end{enumerate}
 

\end{rSection}



%----------------------------------------------------------------------------------------
%	Presentations
%----------------------------------------------------------------------------------------

\begin{rSection}{Invited Talks}
		
	\begin{enumerate}[label=T\arabic* --]
		
		\item \emph{Privacy Preserving Machine Learning.} Arizona State University, Tempe AZ, USA, March 2021.
		
		\item \emph{Efficient Private Set Intersection from Homomorphic Encryption.} Simons Institute, UC Berkeley CA, USA, August 2020.
		
		\item \emph{Endemic Oblivious Transfer.} Sanford, Palo Alto CA, USA, August 2019.
		
		\item \emph{Improved Private Set Intersection.} Google, New York NY, USA, December 2017.
		
		\item  \emph{Fast Private Set Intersection from Homomorphic Encryption.} MIT, Boston Massachusetts, December 2017.
		
		\item \emph{A Survey of Oblivious RAM Methods and Optimizations.} Intel seminar, Hillsboro OR, USA, March 2015.
		
	\end{enumerate}
	
\end{rSection}


%----------------------------------------------------------------------------------------
%	Software Projects
%----------------------------------------------------------------------------------------

\begin{rSection}{Software Projects}
	
	
	\begin{enumerate}[label=S\arabic* --]
		
		\item Hao Chen, Kim Laine and Peter Rindal. \emph{Asymmetric Private Set Intersection} (Microsoft).
		
		\item Peter Rindal. \emph{SMILY: Secure Multi-party Computation Library.} (Microsoft).
		
		\item  Peter Rindal. \emph{$ABY^3$: A Mixed Protocol Framework for Machine Learning.}  (Visa)
		
		
		\item Peter Rindal. \emph{Threshold Authenticated Encryption.} (Visa)
		
		\item Peter Rindal. \emph{Hydra: Threshold ECDSA.} (Visa)
		
		\item Peter Rindal. \emph{libOTe: A fast, portable, and easy to use Oblivious Transfer Library.}  (Open Source) \url{https://github.com/osu-crypto/libOTe}. Includes the protocols of:
		\begin{itemize}
			\item Semi-honest 1-out-of-2 OT [IKNP03]. 
			\item Semi-honest 1-out-of-N OT [KKRT16].
			\item Malicious 1-out-of-2 OT [KOS15].
			\item Malicious 1-out-of-2 Delta-OT [KOS15],[BLNNOOSS15].
			\item Malicious 1-out-of-N OT [OOS16].
			\item Malicious approximate K-out-of-N OT [RR17].
			\item Malicious 1-out-of-2 base OT [NP00].
		\end{itemize}
		\item Peter Rindal. \emph{Ivory-Runtime: A generic Secure Computation API for garbled circuits, SPDZ, etc.} (Open Source)  \url{https://github.com/ladnir/Ivory-Runtime}. Includes the protocols of:
		\begin{itemize}
			\item Semi-honest 2PC [Yao82],[ZRE14].
			\item Semi-honest 3PC [FLNW16].
		\end{itemize}
		
		\item Peter Rindal and Ni Ni Triue. \emph{libPSI: A library for malicious and semi-honest Private Set Intersection.} (Open Source) \url{https://github.com/osu-crypto/libPSI}. Includes the protocols of:
		\begin{itemize}		
		\item Semi-honest Bloom filter PSI [DCW10]. 
		\item Semi-honest cuckoo hashing PSI [KKRT16]. 
		\item Malicious Bloom filter PSI [RR17a]. 
		\item Malicious public key crypto PSI [DKT10]. 
		\item Malicious cuckoo hashing PSI [RR17b]. 
		\item Semi-honest PIR [BGI16].		
	\end{itemize}

		\item Peter Rindal and Roberto Trifiletti. \emph{SplitCommit: A portable C++ implementation of the [FJNT16] XOR-homomorphic commitment scheme.} (Open Source) \url{https://github.com/AarhusCrypto/SplitCommit}
		
		\item Peter Rindal. \emph{Batch Dual Execution: Malicious secure online/offline MPC implementation.}  (Open Source) \url{https://github.com/osu-crypto/batchDualEx}
	\end{enumerate}
	
\end{rSection}



%----------------------------------------------------------------------------------------
%	Service
%----------------------------------------------------------------------------------------

\begin{rSection}{Service}
	
	External reviewer:
	\begin{enumerate}[label=E\arabic* --]
		
		\item \emph{15th Theory of Cryptography Conference (TCC 2017).}  Baltimore MD, USA on November, 2017.
		
		\item \emph{2nd IEEE European Symposium on Security and Privacy (EuroS\&P 2017).} Paris France on April 2017.
		
		\item \emph{19th International Symposium on Stabilization, Safety, and Security of Distributed Systems (SSS 2017).} Boston, Massachusetts, USA on November, 2017.
		
		\item \emph{ 18th International Conference on Cryptology in India (Indocrypt 2017)}. Chennai India on December 2017.
		
		\item \emph{39th IEEE Symposium on Security and Privacy (S\&P 2018)}. San Francisco California, USA on May 2018.
		
		\item \emph{21st edition of the International Conference on Practice and Theory of Public Key Cryptography (PKC 2018)}. Rio De Janeiro, Brazi on March 2018.
		
		
		\item \emph{ 38th International Cryptology Conference (Crypto 2018).} Santa Barbara California, USA on August 2018.
		
		\item \emph{DBSec 2018 : 32nd IFIP WG 11.3 Conference on Data and Applications Security and Privacy}.  Bergamo, Italy on July 2018.
		
	\end{enumerate}
	
\end{rSection}



%----------------------------------------------------------------------------------------
%	Service
%----------------------------------------------------------------------------------------

\begin{rSection}{References}
	
	\begin{enumerate}[label=R\arabic* --]
		
		\item Mike Rosulek, \emph{Principle Ph.D. Advisor.} rosulekm@eecs.oregonstate.edu
		
		\item Payman Mohassel,  \emph{Visa Research Manager. Now at Facebook.} payman.mohassel@gmail.com 
		
		\item Hoe Chen, \emph{Microsoft Research Coworker.  Now at Facebook.}
		
		
	\end{enumerate}
	
\end{rSection}


%----------------------------------------------------------------------------------------
%	EXAMPLE SECTION
%----------------------------------------------------------------------------------------

%\begin{rSection}{Section Name}

%Section content\ldots

%\end{rSection}

%----------------------------------------------------------------------------------------

\end{document}
